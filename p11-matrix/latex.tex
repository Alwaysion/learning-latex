\documentclass{ctexart}
\usepackage{amsmath}

\newcommand{\adots}{\mathinner{
    \mkern2mu
    \raisebox{0.1em}{.}\mkern2mu\raisebox{0.4em}{.}
    \mkern2mu\raisebox{0.7em}{.}\mkern1mu}
    }


\begin{document}
    % 矩阵环境,用&分隔列,用\\分隔行
    \[
        \begin{matrix}
            0 & 1   \\
            1 & 0
        \end{matrix} \qquad
        % pmatrix环境
        \begin{pmatrix}
            0 & -i  \\
            i & 0
        \end{pmatrix} \qquad
        % bmatrix环境
        \begin{bmatrix}
            0 & -1  \\
            1 & 0
        \end{bmatrix} \qquad
        % Bmatrix环境
        \begin{Bmatrix}
            1 & 0   \\
            0 & -1
        \end{Bmatrix} \qquad
        % vmatrix环境
        \begin{vmatrix}
            a & b   \\
            c & d
        \end{vmatrix} \qquad
        % Vmatrix环境
        \begin{Vmatrix}
            i & 0   \\
            0 & -i
        \end{Vmatrix}
    \]

    % 可以使用上下标
    \[
        A = \begin{pmatrix}
            a_{11}^2 & a_{12}^2 & a_{13}^2  \\
            0        & a_{22}   & a_{23}    \\
            0        & 0        & a_{33}
        \end{pmatrix}
    \]

    % 常用省略号:\dots、\vdots、\ddots、\adots
    \[
        A = \begin{bmatrix}
            a_{11} & \dots  & a_{1n}    \\
            \adots & \ddots & \vdots    \\
            0      &        & a_{nn}
        \end{bmatrix}_{n \times n}
    \]

    % 分块矩阵(矩阵嵌套)
    \[
        \begin{pmatrix}
            \begin{matrix}
                1 & 0   \\
                0 & 1
            \end{matrix}
            &
            \text{\Large 0}   \\

            \text{\Large 0}
            &
            \begin{matrix}
                1 & 0   \\
                0 & -1
            \end{matrix}
        \end{pmatrix}
    \]

    % 三角矩阵
    \[
        \begin{pmatrix}
            a_{11}                                              & a_{12} & \cdots & a_{1n}  \\
                                                                & a_{22} & \cdots & a_{2n}  \\
                                                                &        & \ddots & \vdots  \\
            \multicolumn{2}{c}{\raisebox{1.3ex}[0pt]{\Huge 0}}  & & a_{nn}
        \end{pmatrix}
    \]

    % 跨列的省略号:\hdotsfor{<列数>}
    \[
        \begin{pmatrix}
            1 & \frac 12 & \dots & \frac 1n \\
            \hdotsfor{4}    \\
            m & \frac m2 & \dots & \frac mn
        \end{pmatrix}
    \]

    % 行内小矩阵(smallmatrix)环境
    复数$z = (x, y)$也可以用矩阵
    \begin{math}
        \left(  % 要手动加上左括号和右括号
            \begin{smallmatrix}
                x & -y \\
                y & x
            \end{smallmatrix}
        \right)
    \end{math}来表示。

    % array环境(类似于表格环境tabular)
    \[
        \begin{array}{r|r}
            \frac 12    & 0 \\
            \hline
            0           & -\frac a{bc}
        \end{array}
    \]

    % 用array环境构造复杂矩阵
    \[
        % @{<内容>} 添加任意内容,不占表项计数
        % 此处添加一个负值空白,表示向左移-5pt的距离
        \begin{array}{c@{\hspace{-5pt}}l}
            % 第1行,第1列
            \left(
                \begin{array}{ccc|ccc}
                    a   & \cdots & a                                                      & b         & \cdots & b      \\
                        & \ddots & \vdots                                                 & \vdots    & \adots          \\
                        &        & a                                                      & b                           \\ \hline

                        &        &                                                        & c         & \cdots  & c     \\
                        &        &                                                        & \vdots    &        & \vdots \\
                    \multicolumn{3}{c|}{\raisebox{2ex}[0pt]{\Huge 0}}                     & c      & \cdots & c
                \end{array}
            \right)
            &

            % 第1行第2列
            \begin{array}{l}
                % \left.仅表示与\right配对,什么都不输出
                \left.\rule{0mm}{7mm}\right\}p \\
                \\
                \left.\rule{0mm}{7mm}\right\}q
            \end{array}
            \\[-5pt]

            % 第2行第1列
            \begin{array}{cc}
                \underbrace{\rule{17mm}{0mm}}_{m} & \underbrace{\rule{17mm}{0mm}}_{m}
            \end{array}

            & % 第2行第2列
        \end{array}
    \]
    
\end{document}