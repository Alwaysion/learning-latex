\documentclass{ctexart}
\usepackage{hyperref}  % 这里加入此宏包原本只是为了在生成的PDF文件里加入超链接,后来发现竟然可以实现”点击引用括号里的数字可以跳转到对应参考文献条目“的效果
% \usepackage[round]{natbib}
% 可以使用更多的样式,如plainnat
% round可选参数可以将方括号编程圆括号
% 该宏包还提供了\citet和\citep命令

\hypersetup{
    colorlinks=true,
    linkcolor=blue,
    filecolor=blue,
    urlcolor=blue,
    citecolor=cyan,
}

\bibliographystyle{plain}  % 其它样式还有unsrt,alpha,abbrv等


\begin{document}
    % 一、直接把参考文献写在LaTeX源文件里
    % 一次管理,一次使用
    % 参考文献格式
    % \begin{thebibliography}{编号样本}
    %     \bibitem[记号]{引用标志}文献条目1
    %     \bibitem[记号]{引用标志}文献条目2
    %     ...
    % \end{thebibliography}
    % 其中文献条目包括:作者,题目,出版社,年代,版本,页码等。
    % 引用时候要采用:\cite{引用标志1,引用标志2,...}

    引用一篇文章\cite{article1} 引用一本书\cite{book1} 引用另一本书\cite{latexGuide}等等
    \begin{thebibliography}{99}
        \bibitem{article1}陈立辉, 苏伟, 蔡川, 陈晓云. \emph{基于\LaTeX 的Web数学公式提取方法研究}[J]. 计算机科学. 2014(06)
        \bibitem{book1}William H. Press, Saul A. Teukolsky, William T. Vetterling, Brian P. Flanery. \emph{Numerical Recipes 3rd Edition: The Art of Scientific Computing}. Cambridge University Press, New York, 2007.
        \bibitem{latexGuide}Kopka Helmut, W. Daly Patrick. \emph{Guide to \LaTeX}, $4^{th}$ Edition. Available at \texttt{https://www.amazon.com}.
        \bibitem{latexMath}Grätzer George. \emph{Math Into \LaTeX}. Birkhäuser Boston; 3 edition (June 22, 2000).
    \end{thebibliography}

    ----------------------------------------手动分隔线------------------------------------

    % 二、用外部文件管理参考文献
    % 一次管理,多次使用
    上面第4条参考文献是一本书,但格式似乎不对,应该是老师直接用了书的信息(见:\href{https://www.amazon.com/Math-LaTeX-George-Gr%C3%A4tzer-2000-06-22/dp/B017YC6E06}{Math Into LaTeX by George Grätzer (2000-06-22): Amazon.com: Books})
    
    从下面第1条参考文献\cite{latexMath2}可见,正确的书籍引用文献条目格式大致应该为:作者. 标题. 出版社, 地址, 版本, 年份.(另见文件latexMath.bib)(貌似文献引用格式bittex中,Month少见...)

    这是一个参考文献的引用:\cite{mittelbach2004}  % 如果下面的\bibliography命令中没添加该引用文献定义所在的bib文件的文件名(test),这里将会显示成问号而不是数字

    这是另一个引用:\cite{patashnik1988designing}

    \nocite{*}  % 加上这行可以使未被引用的参考文献也出现在“参考文献”中。如果文档中仅有参考文献而未有任何引用,且不加上这行,bibtex编译器会报错(也从视频中可见)
    \bibliography{latexMath, test, cnki}
    % 也可以把所有参考文献都放在一个bib文件中来管理

    Zotero下载:\href{https://www.zotero.org/download/}{Zotero | Downloads}

    值得注意的是,现在0202年的Zotero不像视频中的那样安装浏览器插件就行,还要安装桌面版,而插件则变成了Zotero Connector,顾名思义,用来沟通桌面版Zetero的

\end{document}