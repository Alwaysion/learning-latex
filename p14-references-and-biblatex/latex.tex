\documentclass{ctexart}
\usepackage{hyperref}

\hypersetup{
    colorlinks=true,
    linkcolor=blue,
    filecolor=blue,
    urlcolor=blue,
    citecolor=cyan,
}

% biblatex/biber
% 新的TeX参考文献排版引擎
% 样式文件(参考文献样式文件--bbx文件,引用样式文件--cbx文件)使用LaTeX编写(编写和维护相对简单)
% 支持根据本地化排版,如:
%    biber -l zh__pinyin texfile,按拼音排序
%    biber -l zh__stroke texfile,按笔画排序

% \usepackage[style = numeric, backend = biber]{biblatex}
\usepackage[style = caspervector, backend = biber, utf8, sorting = cenyt]{biblatex}
% v0.3.0开始,原 centy/ecnty 排序方案被 cenyt/ecnyt 排序方案取代
% c,Chinese;e,English;n,name;y,year;t,title

\addbibresource{test.bib}  % 不能省略后缀名


\begin{document}
    % 一次管理,多次应用
    % 无格式化引用\cite{biblatex}  % 使用这个下载的caspervector样式时,如果引用了bib数据库中未定义的参考文献,则会导致参考文献编号全为零

    带方括号引用\parencite{3-2}

    上标引用\supercite{6-1}

    \nocite{*}
    % \printbibliography[title = 参考文献]
    \printbibliography[title = 文献] % 使用caspervector样式时,title默认值为“参考文献”

    biblatex-caspervector:\href{https://github.com/CasperVector/biblatex-caspervector}{CasperVector/biblatex-caspervector: A simple, nice and easily extensible biblography / citation style for Chinese LaTeX users}

\end{document}