\documentclass{ctexart}

% \newcommand 定义命令
% 命令只能有字母组成,不能以\end开头
% \newcommand{<命令>}[<参数个数>][<首参数默认值>]{<具体定义>}

\newcommand{\PRC}{People's Republic of \emph{China}}

% \newcommand也可以使用参数
% 参数个数可以从1到9,使用时用 #1, #2, ..., #9表示
\newcommand{\loves}[2]{#1 喜欢 #2}
\newcommand{\hatedby}[2]{#2 不受 #1 喜欢}

% \newcommand的参数也可以有默认值
% 指定参数个数的同时指定了首个参数的默认值,那么这个命令的
% 第一个参数就成为可选的参数(要使用中括号指定)
\newcommand{\love}[3][喜欢]{#2#1#3}

% \renewcommand 重新定义命令
% 与\newcommand命令作用和用法相同,但只能用于已有命令
% \renewcommand{<命令>}[<参数个数>][<首参数默认值>]{<具体定义>}
\renewcommand{\abstractname}{简介}

% 定义和重新定义环境
% \newenvironment{<环境名称>}[<参数个数>][<首参数默认值>]{<开始定义>}{<结束定义>}
% \renewenvironment{<环境名称>}[<参数个数>][<首参数默认值>]{<开始定义>}{<结束定义>}

% 为book类定义摘要(abstract)环境
\newenvironment{myabstract}[1][摘要]{
    \small
    \begin{center}
        \bfseries #1
    \end{center}
    \begin{quotation}
}{
    \end{quotation}
}

% 环境参数只有<开始定义>中可以使用参数,
% <结束定义>中不能再使用环境参数
% 如果有需要,可以先把前面得到参数保存在一个命令中,在后面使用:
\newenvironment{Quotation}[1]{
    \newcommand{\quotesource}{#1}
    \begin{quotation}
}{
    \par\hfill---\textit{\quotesource}
    \end{quotation}
}


\begin{document}
    \PRC

    \loves{猫儿}{鱼}

    \hatedby{猫儿}{萝卜}
    
    \love[最爱]{猫儿}{鱼}

    \begin{abstract}
        这是一段摘要...
    \end{abstract}

    \begin{myabstract}[我的摘要]
        这是一段自定义格式的摘要...
    \end{myabstract}

    \begin{Quotation}{易$\cdot$乾}
        初九,潜龙勿用
    \end{Quotation}

    定义命令和环境是进行\LaTeX{}格式定制、达成内容与格式分离目标的利器。使用自定义的命令和环境把字体、字号、缩进、对齐、间距等各种琐细的内容包装起来,赋予一个有意义的名字,可以是文档结构清晰、代码整洁、易于维护。在使用宏定义的功能是,要综合利用各种以后命令、环境、变量等功能,事实上,前面所介绍的长度变量与盒子、字体字号等内容,大多并不能直接出现在文档正文中,而主要都是用在实现各种结构化的宏定义里。
    
\end{document}