\documentclass{article}
\usepackage{ctex}  % 使用中文宏包;也可以把文档类花括号的参数改为ctexart,然后把该行注释掉,可以达到同样效果;另外还有,ctexbook,ctexrep(没有与letter对应的

\usepackage{hyperref}

\newcommand\degree{^\circ}

\title{\heiti LaTeX笔记}  % 可以指定字体
\author{\kaishu 千魂剑}
\date{\today}

\hypersetup{
    colorlinks=true,
    linkcolor=blue,
    filecolor=blue,
    urlcolor=blue,
    citecolor=cyan,
}


\begin{document}
    \maketitle
    勾股定理可以用现代语言表述如下:

    直角三角形斜边的平方等于两腰的平方和。

    % 注意:\angle和C之间要一个空格
    可以用符号语言表述为:设直角三角形$ABC$,其中$\angle C=90\degree$,则有:
    \begin{equation}  % 这个equation环境产生带编号的行间公式
        AB^2 = BC^2 + AC^2
    \end{equation}
    
    \CTeX 文档:\href{https://texdoc.net/texmf-dist/doc/latex/ctex/ctex.pdf}{CTeX 宏集手册 - ctex.pdf}
    
    \LaTeXe 文档:\href{https://texdoc.net/texmf-dist/doc/latex/lshort-chinese/lshort-zh-cn.pdf}{The Short Introduction to LaTeX2e (Chinese Simplified) - lshort-zh-cn.pdf}

\end{document}