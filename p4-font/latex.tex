\documentclass[12pt]{article}  % 方括号为字体大小(nomalsize),是可选参数,一般只有10,11和12
\usepackage{ctex}

\newcommand{\myfont}{\textit{\textbf{\textsf{Fancy Text}}}}  % LaTeX讲求内容与形式分离的思想,要修改字体样式仅修改此处即可


\begin{document}
    % LaTeX中字体有五种属性,字体编码(正文、数学),字体族(罗马,无衬线、打字机),字体系列(粗细、宽度),字体形状(直立、斜体、伪斜体、小型大写),字体大小
    
    % 字体族:
    \textrm{Roman Family} \textsf{Sans Serif Family} \texttt{Typewriter Family}  % 和下面效果相同

    {\rmfamily Roman Family} {\sffamily Sans Serif Family} {\ttfamily Typewriter Family}

    {\sffamily who you are? you find self on everyone around. take you as the same as others!}
    % 如果上面这行不加括号的话,字体声明会一直影响后续字体,直到遇到新的声明
    % 很多弹幕说下面中文字体的粗体或斜体不生效,估计是上面没加括号的原因

    {\ttfamily Are you wiser than others? definitely no. in some ways, may it is true. What can you achieve? a luxurious house? a brilliant car? an admirable career? who knows?}

    % 字体系列:
    \textup{Upright Shape} \textit{Italic Shape} \textsl{Slanted Shape} \textsc{Small Caps Shape}

    {\upshape Upright Shape} {\itshape Italic Shape} {\slshape Slanted Shape} {\scshape Small Caps Shape}

    % 中文字体:
    {\songti 宋体} \quad {\heiti 黑体} \quad {\fangsong 仿宋} \quad {\kaishu 楷书}

    中文字体的\textbf{粗体}与\textit{斜体}  % 中文斜体用楷书表示

    % 字体大小(与nomalsiaze相对的大小):
    {\tiny          Hello}\\
    {\scriptsize    Hello}\\
    {\footnotesize  Hello}\\
    {\small         Hello}\\
    {\normalsize    Hello}\\
    {\large         Hello}\\
    {\Large         Hello}\\
    {\LARGE         Hello}\\
    {\huge          Hello}\\
    {\Huge          Hello}

    % 中文字号设置命令:
    \zihao{-0} 你好!  % -0(负零)表示小初号

    \zihao{5} 你好!  % 也可以为5

    \myfont
    
\end{document}